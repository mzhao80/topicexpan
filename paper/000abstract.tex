\reducetxt{Topic taxonomies display hierarchical topic structures of a text corpus and provide topical knowledge to enhance various NLP applications.
To dynamically incorporate new topic information, several recent studies have tried to expand (or complete) a topic taxonomy by inserting emerging topics identified in a set of new documents.
However, existing methods focus only on frequent terms in documents and the local topic-subtopic relations in a taxonomy, which leads to limited topic term coverage and fails to model the global topic hierarchy.
% However, topic taxonomies obtained by existing methods have low coverage of various topic terms and fail to represent consistent topic relations.
% This is because their process of discovering novel topics focuses on only first-order topic relations (i.e. topic-subtopic) and a limited number of candidate terms which are extracted based on term frequency in the given corpus.
In this work, we propose a novel framework for topic taxonomy expansion, named \proposed, which directly generates topic-related terms belonging to new topics.
Specifically, \proposed leverages the hierarchical relation structure surrounding a new topic and the textual content of an input document for topic term generation.
This approach encourages newly-inserted topics to further cover important but less frequent terms as well as to keep their relation consistency within the taxonomy.
% To this end, \proposed first trains a topic-conditional term generator that captures the interaction among a topic, a document, and a topic-related term.
% Then, it identifies new topics by generating the terms that should belong to a virtual topic which is assumed to be located at each valid position in the topic hierarchy.
Experimental results on two real-world text corpora show that \proposed significantly outperforms other baseline methods in terms of the quality of output taxonomies.}