% This must be in the first 5 lines to tell arXiv to use pdfLaTeX, which is strongly recommended.
\pdfoutput=1
% In particular, the hyperref package requires pdfLaTeX in order to break URLs across lines.

\documentclass[11pt]{article}

% Remove the "review" option to generate the final version.
% \usepackage[review]{EMNLP2022}
\usepackage{EMNLP2022}

% Standard package includes
\usepackage{times}
\usepackage{latexsym}

% User defined packages
\usepackage{bm}
\usepackage{mathtools}
\usepackage{float}
\usepackage[ruled,vlined,linesnumbered]{algorithm2e}
\usepackage{algpseudocode}
\usepackage{leftidx}
\usepackage{url}
\usepackage{tabularx, multirow, multicol}
\usepackage{color, colortbl}
\usepackage{soul}
\usepackage{caption}
\usepackage{subcaption}
\usepackage{float}
\usepackage{balance}
\usepackage{setspace}
\usepackage{bbm}
\usepackage{pifont}
\usepackage{microtype}
\usepackage[normalem]{ulem}
\usepackage{wasysym}

\usepackage{amsmath}
\usepackage{amsfonts}
\usepackage{booktabs}
\usepackage{graphicx}
\usepackage{dsfont}
\usepackage{xspace}
\usepackage{hyperref}

\definecolor{Gray}{gray}{0.95}

\newcolumntype{B}{>{\raggedright\arraybackslash}m{0.7\linewidth}}
\newcolumntype{P}{>{\centering\arraybackslash}m{0.16\linewidth}}
\newcolumntype{Q}{>{\raggedright\arraybackslash}m{0.6\linewidth}}
\newcolumntype{R}{>{\raggedright\arraybackslash}m{0.71\linewidth}}
\newcolumntype{T}{>{\raggedright\arraybackslash}m{0.61\linewidth}}
\newcolumntype{U}{>{\centering\arraybackslash}m{0.23\linewidth}}
\newcolumntype{V}{>{\centering\arraybackslash}m{0.18\linewidth}}

\newcolumntype{S}{>{\raggedright\arraybackslash}m{0.5\linewidth}}
\newcolumntype{W}{>{\raggedright\arraybackslash}m{0.55\linewidth}}

\setlength{\abovecaptionskip}{5pt}
\setlength{\belowcaptionskip}{5pt}
\setlength{\floatsep}{10pt plus 1.0pt minus 1.0pt}
\setlength{\textfloatsep}{10pt plus 1.0pt minus 1.0pt}
\setlength{\dblfloatsep}{10pt plus 1.0pt minus 1.0pt}
\setlength{\dbltextfloatsep}{10pt plus 1.0pt minus 1.0pt}

% For proper rendering and hyphenation of words containing Latin characters (including in bib files)
\usepackage[T1]{fontenc}
% For Vietnamese characters
% \usepackage[T5]{fontenc}
% See https://www.latex-project.org/help/documentation/encguide.pdf for other character sets

% This assumes your files are encoded as UTF8
\usepackage[utf8]{inputenc}

% This is not strictly necessary, and may be commented out.
% However, it will improve the layout of the manuscript,
% and will typically save some space.
\usepackage{microtype}

% This is also not strictly necessary, and may be commented out.
% However, it will improve the aesthetics of text in
% the typewriter font.
\usepackage{inconsolata}

\linespread{0.985}

\input{dfn}

% If the title and author information does not fit in the area allocated, uncomment the following
%
%\setlength\titlebox{<dim>}
%
% and set <dim> to something 5cm or larger.

\title{\reducetxt{Topic Taxonomy Expansion via Hierarchy-Aware Topic Phrase Generation}}

% Author information can be set in various styles:
% For several authors from the same institution:
% \author{Author 1 \and ... \and Author n \\
%         Address line \\ ... \\ Address line}
% if the names do not fit well on one line use
%         Author 1 \\ {\bf Author 2} \\ ... \\ {\bf Author n} \\
% For authors from different institutions:
% \author{Author 1 \\ Address line \\  ... \\ Address line
%         \And  ... \And
%         Author n \\ Address line \\ ... \\ Address line}
% To start a seperate ``row'' of authors use \AND, as in
% \author{Author 1 \\ Address line \\  ... \\ Address line
%         \AND
%         Author 2 \\ Address line \\ ... \\ Address line \And
%         Author 3 \\ Address line \\ ... \\ Address line}

\newcommand*\samethanks[1][\value{footnote}]{\footnotemark[#1]}
% \thanks{ \ \ Corresponding author}

\author{Dongha Lee\textsuperscript{1}\hspace{-3pt}, Jiaming Shen\textsuperscript{2}\hspace{-3pt}, Seonghyeon Lee\textsuperscript{3}\hspace{-3pt}, Susik Yoon\textsuperscript{1}\hspace{-3pt}, Hwanjo Yu\textsuperscript{3}\thanks{\ \ Corresponding author}, and Jiawei Han\textsuperscript{1} \\
  \textsuperscript{1}University of Illinois at Urbana-Champaign (UIUC), Urbana, IL, United States \\
  \textsuperscript{2}Google Research, New York, NY, United States \\
  \textsuperscript{3}Pohang University of Science and Technology (POSTECH), Pohang, Republic of Korea \\
  {\{donghal,susik,hanj\}@illinois.edu, jmshen@google.com, \{sh0416,hwanjoyu\}@postech.ac.kr}\\}
 
 
 %\hypersetup{draft}

\begin{document}
\maketitle
\begin{abstract}
\reducetxt{Topic taxonomies display hierarchical topic structures of a text corpus and provide topical knowledge to enhance various NLP applications.
To dynamically incorporate new topic information, several recent studies have tried to expand (or complete) a topic taxonomy by inserting emerging topics identified in a set of new documents.
However, existing methods focus only on frequent terms in documents and the local topic-subtopic relations in a taxonomy, which leads to limited topic term coverage and fails to model the global topic hierarchy.
% However, topic taxonomies obtained by existing methods have low coverage of various topic terms and fail to represent consistent topic relations.
% This is because their process of discovering novel topics focuses on only first-order topic relations (i.e. topic-subtopic) and a limited number of candidate terms which are extracted based on term frequency in the given corpus.
In this work, we propose a novel framework for topic taxonomy expansion, named \proposed, which directly generates topic-related terms belonging to new topics.
Specifically, \proposed leverages the hierarchical relation structure surrounding a new topic and the textual content of an input document for topic term generation.
This approach encourages newly-inserted topics to further cover important but less frequent terms as well as to keep their relation consistency within the taxonomy.
% To this end, \proposed first trains a topic-conditional term generator that captures the interaction among a topic, a document, and a topic-related term.
% Then, it identifies new topics by generating the terms that should belong to a virtual topic which is assumed to be located at each valid position in the topic hierarchy.
Experimental results on two real-world text corpora show that \proposed significantly outperforms other baseline methods in terms of the quality of output taxonomies.}
\end{abstract}


\section{Introduction}
\label{sec:intro}
Topic taxonomy is a tree-structured representation of hierarchical relationship among multiple topics found in a text corpus~\cite{zhang2018taxogen, shang2020nettaxo, meng2020hierarchical}. 
Each topic node is defined by a set of semantically coherent terms related to a specific topic (i.e., topic term cluster), and each edge implies the ``general-specific'' relation between two topics (i.e., topic-subtopic).
With the knowledge of hierarchical topic structures, topic taxonomies have been successfully utilized in many text mining applications, such as text summarization~\cite{petinot2011hierarchical, bairi2015summarization} and categorization~\cite{meng2019weakly, shen2021taxoclass}.

\begin{figure}[t]
    \centering
    \includegraphics[width=\linewidth]{FIG/problem.pdf}
    \caption{An example of topic taxonomy expansion. The known (i.e., existing) topics and novel topics are in single-line and double-line boxes, respectively.}
    \label{fig:problem}
\end{figure}

Recently, automated expansion (or completion) of an existing topic taxonomy has been studied \cite{huang2020corel, lee2022taxocom}, which helps people to incrementally manage the topic knowledge within fast-growing document collections.
This task has two technical challenges:
% These methods typically consists of two steps:
(1) identifying new topics by collecting topic-related terms that have novel semantics, and (2) inserting the new topics at the right position in the hierarchy.
%, by leveraging the initial taxonomy as additional supervision. 
In Figure~\ref{fig:problem}, for example, a new topic node \textit{painter} that consists of its topic-related terms [\textit{baroque painter}, \textit{realist painter}, \textit{portraitist}, ...] is inserted at the child position (i.e., subtopic) of the existing topic node \textit{artist}, without breaking the consistency of topic relations with the neighbor nodes.
% For example, in Figure~\ref{fig:problem}, new topic nodes (``cakes'': [``fruit cake'', ``ounce cake'', $\ldots$]) and (``cookies'': [``chocolate cookies'', ``coconut chip cookies'', $\ldots$]) are inserted into the child (i.e., subtopic) of the existing topic node (``breads bakery'': [``pastries'', ``loaf'', $\ldots$]) while keeping the consistency of topic relations with the neighbor nodes.
% The existing methods mainly employ term embedding and clustering techniques~\cite{mikolov2013distributed, meng2020hierarchical} to infer the semantic relevance among terms as well as the hierarchical relation between a topic and its new subtopic.

The existing methods for topic taxonomy expansion, however, suffer from two major limitations:
(1) \textit{Limited term coverage} --
They identify new topics from a set of candidate terms, while relying on entity extraction tools~\cite{zeng2020tri} or phrase mining techniques~\cite{liu2015mining,shang2018automated,gu2021ucphrase} to obtain the high-frequency candidate terms in a corpus.
Such extraction techniques will miss a lot of topic-related terms that have low frequency, and thus lead to an incomplete set of candidate terms~\cite{zeng2021enhancing}.
(2) \textit{Inconsistent topic relation} --
As they insert new topics by considering only the first-order relation between two topics (i.e., a topic and its subtopic), the newly-inserted topics are likely to have inconsistent relations with other existing topics.
The expansion strategy based on the first-order topic relation is inadequate to capture the holistic structure information of the existing topic taxonomy.
%, which eventually hinders new topic nodes from being inserted into the right position in the topic hierarchy
% If a language model can encodes topic knowledge as well as their hierarchical relationship, it can generate a from any input document.

As a solution to both challenges, we present \proposed, a new framework that expands the topic taxonomy via \textit{hierarchy-aware topic term generation}.
The key idea is to directly generate topic-related terms from documents by taking the topic hierarchy into consideration.
% That is, we aim to learn a neural text generation model for generating topic-related terms, each of which is regarded as a sequence of word tokens.
From the perspective of term coverage, this generation-based approach can identify more multi-word terms even if they have low frequency in the given corpus~\cite{zeng2021enhancing}, compared to the extraction-based approach only working on the extracted candidate terms that frequently appear in the corpus.
%compared to the extraction-based approach that requires candidate terms to appear frequently enough for identifying them and learning their reliable embeddings.
To combat the challenge of relation inconsistency, we utilize graph neural networks (GNNs) to encode the relation structure surrounding each topic~\cite{kipf2017semi, shen2021taxoclass} and generate topic-related terms conditioned on these relation structure encodings.
% we use the relation structure surrounding each topic as the condition for generating topic-related terms, with the help of graph neural networks (GNNs) that encode the structural information of the topic hierarchy~\cite{kipf2017semi, shen2021taxoclass}.
This allows us to accurately capture a hierarchical structure beyond the first-order relation between two topics.
% For example, in Figure~\ref{fig:problem}, a new topic ``dairy'' can be correctly inserted into the child of ``grocery gourmet food'' by considering its relation not only with the potential parent but also with all types of its neighbors, such as ancestors, siblings, and all descendants of the siblings.

To be specific, \proposed consists of \textit{the training step} and \textit{the expansion step}.
The training step is for optimizing a neural model that topic-conditionally generates a term from an input document.
% The model architecture is designed to capture the semantic interaction among a topic, a document, and a topic-related term.
Technically, for topic-conditional term generation, the model utilizes the relation structure of a topic node as well as the textual content of an input document.
% In the expansion step, \proposed utilizes the trained model to infer the terms that belong to new topics, which are assumed to \textit{virtually exist} in the topic hierarchy.
The expansion step is for discovering novel topics and inserting them into the topic taxonomy.
To this end, \proposed places a \textit{virtual} topic node underneath each existing topic node, and then it generates the topic terms conditioned on the virtual topic by utilizing the trained model.
In the end, it performs clustering on the generated terms to identify multiple novel topics, which are inserted at the position of the virtual topic node.
% To this end, \proposed introduces a \textit{virtual} topic node as a child node of each existing topic node.
% Using each virtual topic node along with all documents, it generates the terms that are likely to be relevant to the new topic, then it finds out multiple novel topics from the clusters of generated terms.
% In the end, the identified new topic nodes are inserted into the position of the virtual node.

% Our comprehensive evaluation on two real-world topic taxonomies along with their corresponding document collections demonstrates that \proposed obtains higher quality of topic taxonomy in terms of relation consistency as well as term coverage, compared to other baseline methods.
% To be specific, topic-conditional term generation of our framework enables to explore a variety of multi-word terms relevant to each topic, whereas extraction-based baselines fail to identify such terms due to the low term frequency and the incompleteness of candidate terms.
% Besides, the hierarchy-awareness of \proposed, induced by graph neural networks, helps to preserve the consistency of an expanded topic taxonomy, by capturing the new relation structure of a virtual topic node.
% For reproducibility, the implementation will be publicly available through the anonymized github repository during the review process.\footnote{https://github.com/topicexpan-author/topicexpan}


\smallsection{Contributions}
The main contributions of this paper can be summarized as follows:
(1) We propose a novel framework for topic taxonomy expansion, which tackles the challenges in topic term coverage and topic relation consistency via hierarchy-aware topic term generation.
(2) We present a neural model to generate a topic-related term from an input document \textit{topic-conditionally} by capturing the hierarchical relation structure surrounding each topic based on GNNs.
(3) Our comprehensive evaluation on two real-world datasets demonstrates that output taxonomies of \proposed show better relation consistency as well as term coverage, compared to that of other baseline methods.
    
% The main contribution of this paper can be summarized as follows.
% \begin{itemize}
%     \item We present a novel framework for topic taxonomy expansion, which discovers new topics at each valid position in a given taxonomy via hierarchy-aware topic term generation.
%     \item Our model architecture is able to generate a multi-word term from an input document \textit{topic-conditionally} by capturing the relation structure of a topic hierarchy based on GNNs.
%     \item Our comprehensive evaluation on two real-world datasets demonstrates that output taxonomies of \proposed show better relation consistency as well as term coverage, compared to that of other baseline methods.
% \end{itemize}

\section{Related Work}
\label{sec:related}
% In this section, we briefly review the literature on two relevant tasks:
% (i) topic taxonomy construction and (ii) keyphrase prediction.

\smallsection{Topic Taxonomy Construction}
\label{subsec:topictaxo}
% The classic approach to hierarchical topic discovery is to estimate probabilistic topic models that describe the generative process of a topic hierarchy~\cite{blei2003hierarchical, mimno2007mixtures}.
% To avoid their computationally-expensive inference algorithm as well as improve the presentation of the knowledge, recent studies focus on a simple tree-structured topic taxonomy whose node is represented by a set of topic-related terms.
To build a topic taxonomy of a given corpus from scratch, the state-of-the-art methods have focused on finding out discriminative term clusters in a hierarchical manner~\cite{zhang2018taxogen, meng2020hierarchical, shang2020nettaxo}. 
%with the help of the advanced embedding and clustering techniques.
Several recent studies have started to enrich and expand an existing topic taxonomy by discovering novel topics from a corpus and inserting them into the taxonomy~\cite{huang2020corel, lee2022taxocom}.
They leverage the initial topic taxonomy as supervision for learning the hierarchical relation among topics.
% To be specific, they infer the new relation between a topic and its novel subtopic to be inserted,
To be specific, they discover new subtopics that should be inserted at the child of each topic,
by using a relation classifier trained on (parent, child) topic pairs~\cite{huang2020corel} or performing novel subtopic clustering~\cite{lee2022taxocom}. 
However, all the methods rely on candidate terms extracted from a corpus and also consider only the first-order relation between two topics, which degrades the term coverage and relation consistency of output topic taxonomies.

\smallsection{GNN-based Taxonomy Expansion}
\label{subsec:taxoexpan}
Recently, there have been several attempts to employ GNNs for expanding a given entity taxonomy~\cite{mao2020octet,shen2020taxoexpan,zeng2021enhancing}.
% \jw{need some careful explanation of topic taxonomy vs. entity taxonomy?}
% Note that they mainly focus on a conventional entity (or term-level) taxonomy, which differs from a topic (or term cluster-level) taxonomy in that its node represents a single entity or term.
Their goal is to figure out the correct position where a new entity should be inserted, by capturing structural information of the taxonomy based on GNNs.
They mainly focus on an entity taxonomy that shows the hierarchical semantic relation among fine-grained entities (or terms), requiring plenty of nodes and edges in a given taxonomy to effectively learn the inter-entity relation.
In contrast, a topic taxonomy represents coarse-grained topics (or high-level concepts) that encode \textit{discriminative term meanings} as well as \textit{term co-occurrences} in documents (Figure~\ref{fig:problem}), which allows its node to correspond to a topic class of documents.
That~is, it is not straightforward to apply such methods to a topic taxonomy with much fewer nodes and edges, and thus how to enrich a topic taxonomy~with GNNs remains an important research question.
% That is, leveraging GNNs for enriching a topic taxonomy has not been studied yet.

\begin{figure*}[t]
    \centering
    \includegraphics[width=\linewidth]{FIG/framework.pdf}
    \caption{The overall process of \proposed. 
    (Left) It trains a unified model via multi-task learning of topic-document similarity prediction and topic-conditional phrase generation.
    (Right) It selectively collects the phrases conditionally-generated for a virtual topic node, and then it identifies multiple novel topics from phrase clusters. % which accordingly expands the topic taxonomy by inserting the new topic nodes.
    }
    \label{fig:framework}
\end{figure*}

\smallsection{Keyphrase Generation}
\label{subsec:kpg}
The task of keyphrase prediction aims to find condensed terms that concisely summarize the primary information of an input document~\cite{liu2020keyphrase}.
% The most dominant approach to this problem is modeling it as the sequence labeling task, which predicts whether each token in an input text is a part of keyphrases or not~\cite{gollapalli2014extracting}.
% Inspired by the great success of pretrained language models~\cite{devlin2019bert, clark2020electra}, some studies take advantage of the BERT architecture by finetuning it for the target task~\cite{liu2020keyphrase, rungta2020transkp}.
%also has gained much attention~\cite{meng2017deep, zhou2021topic} because they can obtain phrases that do not match any contiguous subsequence of an input text, referred to as absent keyphrases.
The state-of-the-art approach to this problem is modeling it as the text generation task, which sequentially generates word tokens of a keyphrase~\cite{meng2017deep, zhou2021topic}.
They adopt neural architectures as a text encoder and decoder, such as an RNN/GRU~\cite{meng2017deep, wang2019topic} and a transformer~\cite{zhou2021topic}. 
Furthermore, several methods have incorporated a neural topic model into the generation process~\cite{wang2019topic, zhou2021topic} to fully utilize the topic information extracted in an unsupervised way.
Despite their effectiveness, none of them has focused on \textit{topic-conditional} generation of keyphrases from a document, as well as \textit{hierarchical} modeling of topic relations.
% Despite their effectiveness, none of them have explicitly modeled the semantic relationship among a document, a topic, and a topic-related phrase, in order to generate the keyphrase of a document while being conditioned on a specific topic.


% \subsection{Hierarchical Multi-label Text Classification}
% \label{subsec:hmtc}
% Hierarchical Multi-label Text Classification (HMTC) aims to maps an input text into multi-label logits.
% \cite{huang2019hierarchical}
% \cite{zhou2020hierarchy}
% \cite{peng2018large}.
% Most 

% In addition to the conventional supervised approach, \taxoclass~\cite{shen2021taxoclass} presented a weakly-supervised framework for the case where only the class hierarchy and unlabeled documents are given.
% However, all the methods are not able to predict the unseen classes for an input document, 

\section{Problem Formulation}
\label{sec:problem}
% In this section, we formally define our target task, named topic taxonomy expansion.
% We first describe the notations about a topic taxonomy, then formally define our target task, named topic taxonomy expansion.

\smallsection{Notations}
A \textit{topic taxonomy} $\taxo=(\cateset, \edgeset)$ is a tree structure about topics, where each node ($\in\cateset$) represents a single conceptual topic and each edge ($\in\edgeset$) implies the hierarchical relation between a topic and its subtopic.
A topic node $\topic{j}\in\cateset$ is described by the set of topic-related terms, denoted by $\topicphs{j}$ (i.e., term cluster for the topic $\topic{j}$), where the most representative term (i.e., center term) serves as the topic name.
Each document $\doc{i}=[\token{i1}, \ldots, \token{iL}]$ and each term\footnote{Note that our proposed approach considers each term as a sequence of word tokens, and this enables us to handle a much larger number of multi-word terms, compared to the case of using a precomputed set of candidate terms.}
$\phrase{k}=[\token{k1}, \ldots, \token{kT}]$ in a given corpus $\docuset$ is the sequence of $L$ and $T$ word tokens in the vocabulary set $v\in\vocaset$, respectively.
Here, each term is regarded as a phrase that consists of one or more word tokens, so the terms ``phrase'' and ``term'' are used interchangeably in this paper.
% \footnote{For notational convenience, we simply use $L$ and $T$ as the maximum length of a document and a term, respectively.}

\smallsection{Problem Definition}
% \begin{definition}[Topic taxonomy expansion]
% \textit{Topic taxonomy expansion} aims to discover new topics and their relevant terms from a given text corpus $\docuset$, and to expand the initial topic taxonomy $\taxo$ by inserting the new topic nodes at the right position in the taxonomy (Figure~\ref{fig:problem}).
Given a text corpus $\docuset$ and an initial topic taxonomy $\taxo$, the task of topic taxonomy expansion aims to discover novel topics by collecting the topic-related terms from $\docuset$ and insert them at the right position in $\taxo$ (Figure~\ref{fig:problem}).
% \end{definition}

% Note that all previous studies on topic taxonomy construction~\cite{zhang2018taxogen, meng2020hierarchical} and expansion~\cite{huang2020corel, lee2022taxocom} require a predefined set of candidate terms.
% Since a high-quality term set is necessary to obtain a high-quality topic taxonomy, they usually extract key terms from the given document corpus based on entity extraction tools~\cite{zeng2020tri} or phrase mining techniques~\cite{liu2015mining, shang2018automated, gu2021ucphrase}.
% Nevertheless, it is practically infeasible for such extraction process that depends on term frequency or part-of-speech tags to find out all single-word and multi-word terms appearing in the corpus. 
% As a result, this limits the term coverage of their output topic taxonomy.
% On the contrary, our framework does not require the term set rather uses a vocabulary of word tokens $\vocaset$ to represent a term as a sequence of word tokens.
% Note that our proposed approach considers each term as a sequence of word tokens, and this enables to handle much larger number of multi-word terms, compared to the case of using a precomputed set of candidate terms.
%, which the extraction-based approach heavily relies on.
% In other words, each term is assumed to be a phrase that consists of one or more word tokens, so we use the terms ``phrase'' and ``term'' interchangeably in this paper.

\section{\proposed: Proposed Framework}
\label{sec:method}
% In this section, we present our topic taxonomy expansion framework, named \proposed, which effectively finds out novel topics from a text corpus via topic phrase generation conditioned on a newly-introduced topic relation structure.

\subsection{Overview}
\label{subsec:overview}
\proposed consists of (1) \textit{the training step} that trains a neural model for generating phrases topic-conditionally from documents (Figure~\ref{fig:framework} Left)
%to learn the semantic interaction among a topic, a document, and a topic-related phrase, given in the initial topic taxonomy and document corpus, 
and (2) \textit{the expansion step} that identifies novel topics for each new position in the taxonomy by using the trained model (Figure~\ref{fig:framework} Right).
%generated for a newly-introduced topic relation structure (i.e., a new position in the topic hierarchy).
The detailed algorithm is described in Section~\ref{subsec:pseudocode}.
%Algorithm~\ref{alg:overview}.
% The overview is illustrated Figure~\ref{fig:framework} and Algorithm~\ref{alg:overview}.
% Please refer to Section~\ref{subsec:pseudocode} for the detailed algorithm.

% Note that the existing methods retrieve novel topic terms while relying on the embedding vectors of the terms extracted from the text corpus.
% On the contrary, our framework directly generates novel topic phrases token-by-token by leveraging both textual information of the documents and structural information of the topic hierarchy.

\smallsection{Training Step}
\label{subsubsec:training}
% In the training step, 
% \proposed aims to train a neural model which reconstructs the given topic taxonomy from the document corpus. Specifically, it 
\proposed optimizes parameters of its neural model to maximize the total likelihood of the initial taxonomy $\taxo$ given the corpus $\docuset$.
%which is described as follows.
\begin{equation}
\small
\label{eq:likelihood}
    \begin{split}
        P(\taxo;\docuset) &= \prod_{\topic{j}\in\cateset} \prod_{\phrase{k}\in\topicphs{j}} P(\phrase{k}|\topic{j};\docuset) \\
        &= \prod_{\topic{j}\in\cateset} \prod_{\phrase{k}\in\topicphs{j}} \sum_{\doc{i}\in\docuset} P(\phrase{k}, \doc{i} |\topic{j}) \\
        %&= \prod_{\topic{j}\in\cateset} \prod_{\doc{i}\in\docuset} \prod_{\phrase{k}\in\topicphs{j}\cap\doc{i}} P(\phrase{k}, \doc{i} |\topic{j}) \\
        &\approx \prod_{\topic{j}\in\cateset} \prod_{\doc{i}\in\docuset} \prod_{\phrase{k}\in\topicphs{j}\cap\doc{i}} P(\phrase{k}|\doc{i},\topic{j}) P(\doc{i}|\topic{j}).
    \end{split}
    \raisetag{49pt}
\end{equation}
In the end, the total likelihood is factorized into the topic-conditional likelihoods of a document and a phrase, i.e., $P(\doc{i}|\topic{j})$ and $P(\phrase{k}|\doc{i},\topic{j})$, for all the positive triples $(\topic{j}, \doc{i}, \phrase{k})$ collected from $\taxo$ and $\docuset$.
That is, each triple satisfies the condition that its phrase $\phrase{k}$ belongs to the topic $c_j$ (i.e., $p_k\in\topicphs{j}$) and also appears in the document $d_i$.

To maximize Equation~\eqref{eq:likelihood}, we propose a unified model for estimating $P(\doc{i}|\topic{j})$ and $P(\phrase{k}|\doc{i},\topic{j})$ via the tasks of \textit{topic-document similarity prediction} and \textit{topic-conditional phrase generation}, respectively.
% To be specific, the former task increases the similarity between a topic $\topic{j}$ and a document $\doc{i}$, indicating how confidently the document includes any sentences or mentions about the topic.
In Figure~\ref{fig:framework} Left, for each positive triple $(\topic{j}, \doc{i}, \phrase{k})$, the former task increases the similarity between the topic $\topic{j}$ and the document $\doc{i}$.
This similarity indicates how confidently the document $\doc{i}$ includes any sentences or mentions about the topic $\topic{j}$.
% Since we do not have negatively-labeled topic-document pairs, we use randomly sampled documents as negative for discriminative learning of the topic-document similarity.
At the same time, the latter task maximizes the decoding probability of the phrase $\phrase{k}$ (i.e., generates the phrase) 
conditioned on the topic $\topic{j}$ and the document $\doc{i}$.
% by using the relation structure of the topic $\topic{j}$ and the textual content of the document $\doc{i}$ as the condition for generation.
The model parameters are jointly optimized for the two tasks, and each of them will be discussed in Section~\ref{subsec:training}.
% : $\totalloss = \simloss + \genloss$.
% \begin{equation}
%     \totalloss = \simloss + \genloss.
% \end{equation}

\smallsection{Expansion Step}
\label{subsubsec:expansion}
% In the expansion step, 
% \proposed utilizes the trained model along with the document corpus and the topic hierarchy, to find novel topics that should be inserted into the taxonomy.
\proposed expands the topic taxonomy by discovering novel topics and inserting them into the taxonomy.
To this end, it utilizes the trained model to generate the phrases $\phrase{}$ that have a high topic-conditional likelihood $P(\phrase{}|\vtopic{};\docuset)$ for a new topic $\vtopic{}$ from a given corpus $\docuset$.
In Figure~\ref{fig:framework} Right, it first places a virtual topic node $\vtopic{j}$ at a \textit{valid} insertion position in the hierarchy (i.e., a child position of a topic node $\topic{j}$), and then it collects the phrases relevant to the virtual topic by generating them from documents $\doc{i}\in\docuset$.
%whose topic-document similarity is larger than a threshold.
% By doing so, it can collect confident phrases for the virtual topic node.
Finally, it identifies multiple novel topics by clustering the collected phrases into semantically coherent but distinguishable clusters, which are inserted as the new topic nodes at the position of the virtual node.
% so that each of the phrase clusters is semantically coherent but distinguishable from the others, thereby inserting the new topic nodes at the position of the virtual node.
% Finally, it performs clustering on the phrases generated for the virtual topic node to identify multiple novel topics, 
The details will be presented in Section~\ref{subsec:expansion}.
% \jw{need linking to Figures 2 and 3 and give more high-level explanation/illustration of your methods in section 4.}

\subsection{Encoder Architectures}
\label{subsec:training}
For modeling the two likelihoods $P(\doc{i}|\topic{j})$ and $P(\phrase{k}|\doc{i},\topic{j})$, we introduce a topic encoder and a document encoder, which respectively computes the representation of a topic $\topic{j}$ and a document $\doc{i}$.
% (i) The encoder should be \textit{generalizable} to any unseen topics so as to estimate the likelihood of a phrase or a document given a novel topic.
% However, a naive topic encoder that utilizes only topic-ids (e.g., an embedding matrix) or topic names (e.g., pretrained word vectors) as the input features is not capable of obtaining the representation of a novel topic, which is not observable during the training. 
% To this end, \proposed adopts a \textit{structure encoder} that captures the structural information of the topic hierarchy into the topic representations, by taking the local relation graph surrounding each topic as the input features.

\begin{figure}[t]
    \centering
    \includegraphics[width=\linewidth]{FIG/topic_encoder.pdf}
    \caption{\reducetxt{The topic encoder architecture.
    It computes topic representations by encoding a topic relation graph.}
    }
    \label{fig:topic_encoder}
\end{figure}

\subsubsection{Topic Encoder}
\label{subsubsec:topic_encoder}
There are two important challenges of designing the architecture of a topic encoder:
(1) The topic encoder should be \textit{hierarchy-aware} so that the representation of each topic can accurately encode the hierarchical relation with its neighbor topics, and
(2) the representation of each topic needs to be \textit{discriminative} so that it can encode semantics distinguishable from that of the sibling topics.
Hence, we adopt graph convolutional networks (GCNs)~\cite{kipf2017semi} to capture the semantic relation structure surrounding each topic.
%(i.e., semantic relations with the neighbor topics).
% Figure~\ref{fig:topic_encoder} shows the topic encoder architecture.

We first construct a topic relation graph $\mathcal{G}$ by enriching the edges of the given hierarchy $\taxo$ to model heterogeneous relations between topics, as shown in Figure~\ref{fig:topic_encoder}.
The graph contains three different types of inter-topic relations:
(1) downward, (2) upward, and (3) sideward.
The downward and upward edges respectively capture the top-down and bottom-up relations (i.e., hierarchy-awareness).
We additionally insert the sideward edges between sibling nodes that have the same parent node.
Unlike the downward and upward edges, the sideward edges pass the information in a negative way to make topic representations discriminative among the sibling topics.
% Particularly, the sideward edges are necessary for novel topic discovery because they encourage a virtual (i.e., newly-introduced) topic node to encode the novel semantic, which is quite different from its known sibling topics, as shown in Figure~\ref{fig:topic_encoder}(b).
%but satisfies the hierarchical relation with its parent topic.
% For example, in Figure~\ref{fig:topic_encoder}(b), a novel subtopic of ``grocery gourmet food'' should be semantically different from ``breads bakery'' and ``beverages''.
The topic representation of $\topic{j}$ at the $m$-th GCN layer is computed by
\begin{equation}
\small
    \hvec{j}^{(m)}= \phi\left(
    \sideset{}{_{(i, j)\in\mathcal{G}}}\sum \alpha_{r(i, j)} \cdot \bm{W}^{(m-1)}_{r(i, j)} \cdot \hvec{i}^{(m-1)}\right),
\end{equation}
where $\phi$ is the activation function, $r(i, j)\in\{\text{down}, \text{up}, \text{side}\}$ represents the relation type of an edge $(i,j)$, and $\alpha$ indicates either positive or negative aggregation according to its relation type; 
i.e., $\alpha_{\text{down}}=\alpha_{\text{up}}=+1$ and $\alpha_{\text{side}}=-1$.
The \glove word vectors~\cite{pennington2014glove} for each topic name are used as the base node features (i.e., $\hvec{j}^{(0)}$) after being averaged for all tokens in the topic name.
Using a stack of $M$ GCN layers, we finally obtain the representation of a \textit{target} topic node $\topic{j}$ (i.e., the topic node that we want to obtain its representation) by $\cvec{j}=\hvec{j}^{(M)}$.

The topic encoder should also be able to obtain the representation of a virtual topic node, whose topic name is not determined yet, during the expansion step. 
% and it raises a technical challenge that such virtual nodes
% To compute the representation of a target topic even if its name is unknown, 
For this reason, we mask the base node features of a target topic node regardless of whether the node is virtual or not, as depicted in Figures~\ref{fig:topic_encoder}(a) and (b).
% For this reason, we mask the name of a \textit{target} topic node (i.e., the topic node that we want to obtain its representation) with the token \texttt{[MASK]} regardless of whether the node is virtual or not, as illustrated in Figures~\ref{fig:topic_encoder}(a) and (b).
% As a result, the final representation of each topic encodes the relation structure of its $M$-hop neighbor topics not knowing the name of the target topic.
In other words, with the name of a target topic masked, the topic representation encodes the relation structure of its $M$-hop neighbor topics.

\subsubsection{Document Encoder}
\label{subsubsec:document_encoder}
%Among various types of neural architectures that have been used for encoding texts, 
For the document encoder, we employ a pretrained language model, BERT~\cite{devlin2019bert}.
%which is effective to capture the long-term dependency among tokens in an input text.
It models the interaction among the tokens based on the self-attention mechanism, thereby obtaining each token's contextualized representation, denoted by $[\vvec{i1},\ldots,\vvec{iL}]$.
A document representation $\dvec{i}$ is obtained by mean pooling in the end. 

\subsection{Learning Topic Taxonomy}
\label{subsec:training}
In the training step, \proposed optimizes model parameters by using positive triples as training data $\mathcal{X}=\{(\topic{j}, \doc{i}, \phrase{k})|\phrase{k}\in\topicphs{j}\cap\doc{i}, \forall \topic{j}\in\cateset, \forall\doc{i}\in\docuset\}$ 
%to maximize the likelihood (Equation~\eqref{eq:likelihood})
via multi-task learning of {topic-document similarity prediction} and {topic-conditional phrase generation} (Sections~\ref{subsubsec:prediction} and \ref{subsubsec:generation}).
% As discussed in Section~\ref{subsubsec:training}, each positive triple is identified from $\taxo$ and $\docuset$, where a phrase $\phrase{k}$ belongs to a topic $\topic{j}$ and also appears in a document $\doc{i}$ at the same time.  

\subsubsection{Topic-Document Similarity Prediction}
\label{subsubsec:prediction}
The first task is to learn the similarity between a topic and a document.
We define the topic-document similarity score by bilinear interaction between their representations, i.e., $\cvec{j}^\top \bm{M} \dvec{i}$ where $\bm{M}$ is the trainable interaction matrix.
The topic-conditional likelihood of a document in Equation~\eqref{eq:likelihood} is optimized by using this topic-document similarity score, $P(\doc{i}|\topic{j}) \propto \exp(\cvec{j}^\top \bm{M} \dvec{i})$.

The loss function is defined based on InfoNCE \cite{oord2018representation}, which pulls positively-related documents into the topic while pushing away negatively-related documents from the topic.
\begin{equation}
\small
\label{eq:simloss}
    \simloss = - \sum_{(\topic{j},\doc{i},\phrase{k})\in\mathcal{X}} \log \frac{\exp(\cvec{j}^\top\bm{M} \dvec{i}/\gamma)}{\sum_{i'}\exp(\cvec{j}^\top \bm{M} \dvec{i'}/\gamma)},
\end{equation}
where $\gamma$ is the temperature parameter.
For each triple $(\topic{j}, \doc{i}, \phrase{k})$, we use its document $\doc{i}$ as positive and regard documents from all the other triples in the current mini-batch as negatives.

\subsubsection{Topic-Conditional Phrase Generation}
\label{subsubsec:generation}
The second task is to generate phrases from a document being conditioned on a topic.
For the phrase generator, we employ the architecture of the transformer decoder~\cite{vaswani2017attention}.
%which has been widely used in many text generation tasks, such as keyphrase generation~\cite{liu2020keyphrase} text summarization~\cite{zhang2020pegasus}, and machine translation~\cite{vaswani2017attention}

For topic-conditional phrase generation, the \textit{context} representation, $\bm{Q}(\topic{j}, \doc{i})$, needs to be modeled by fusing the textual content of a document $\doc{i}$ as well as the relation structure of a topic $\topic{j}$.
To leverage the textual features while focusing on the topic-relevant tokens, we compute \textit{topic-attentive} token representations and pass them as the input context of the transformer decoder.
%(Figure~\ref{fig:phrase_generator}).
Precisely, the topic-attention score of the $l$-th token in the document $\doc{i}$, $\beta_l(\topic{j},\doc{i})$, is defined by its similarity with the topic.
% , then it is multiplied to the contextualized token representation $\vvec{il}$.
% That is, \textit{topic-attentive document context} is computed as follows.
\vspace{-15pt}
\begin{equation}
\small
\label{eq:gencontext}
    \begin{split}
       \beta_l(\topic{j},\doc{i}) &= {\exp(\cvec{j}^\top\bm{M}\vvec{il})}/{\sideset{}{_{l'=1}^L}\sum\exp(\cvec{j}^\top\bm{M}\vvec{il'})} \\
        \bm{Q}(\topic{j},\doc{i}) &= [\beta_1(\topic{j},\doc{i}) \cdot \vvec{i1}, \ldots, \beta_{L}(\topic{j},\doc{i}) \cdot \vvec{iL}],
    \end{split}
\end{equation}
% It is worth noting
where the interaction matrix $\bm{M}$ is weight-shared with the one in Equation~\eqref{eq:simloss}.
%is weight-shared for computing the topic-attention scores.
%Using the context representation, 
Then, the sequential generation process of a token $\hat{v}_t$ is described by
\begin{equation}
\small
\label{eq:genoutput}
\begin{split}
    \svec{t} &= \text{Decoder}(\hat{v}_{<t};\bm{Q}(\topic{j},\doc{i})) \\ 
    \hat{v}_{t} &\sim \text{Softmax}(\text{FFN}(\svec{t})).
\end{split}
\end{equation}
$\text{FFN}$ is the feed-forward networks for mapping a state vector $\svec{t}$ into vocabulary logits.
Starting from the first token \texttt{[BOP]}, the phrase is acquired by sequentially decoding a next token $\hat{v}_t$ until the last token \texttt{[EOP]} is obtained;
the two special tokens indicate the begin and the end of the phrase.

% \footnote{We adopt the greedy strategy, which selects the word token of the max probability.}
%\footnote{For training, we take the previous tokens from gold standards (i.e., target phrases) as the input token sequence of the transformer decoder, also known as \textit{teacher forcing}.}
% \footnote{For inference, we use the previously predicted tokens as the input of the decoder.}

The loss function is defined by the negative log-likelihood, where the phrase $\phrase{k}=[\token{k1},\ldots,\token{kT}]$ in a positive triple $(\topic{j},\doc{i},\phrase{k})$ is used as the target sequence of word tokens.
\begin{equation}
\small
\label{eq:genloss}
    \begin{split}
        % \genloss &= - \sum_{(\topic{j}, \doc{i}, \phrase{k})} \log P(\phrase{k}|\doc{i},\topic{j}), \\
        \genloss &= - \sum_{(\topic{j}, \doc{i}, \phrase{k})\in\mathcal{X}} \sum_{t=1}^{T}  \ \log P(\token{kt}|\token{k(<t)},\topic{j},\doc{i}). \\
    \end{split}
\end{equation}
% By minimizing the loss, the generator can learn the sequence model for phrases conditioned on a topic-document pair.
% \jw{need a little discussion on why your method can reduce hallucination in phrase generation.}

To sum up, the joint optimization of Equations~\eqref{eq:simloss} and \eqref{eq:genloss} updates all the model parameters in an end-to-end manner, including the similarity predictor, the phrase generator, and both encoders.


\subsection{Expanding Topic Taxonomy}
\label{subsec:expansion}
In the expansion step, \proposed expands the topic taxonomy by utilizing the trained model to generate the phrases 
%having a high topic-conditional likelihood $P(\phrase{}|\vtopic{};\docuset)$ 
for a virtual topic, which is assumed to be located at a valid insertion position in the hierarchy.
For thorough expansion, it considers a child position of every existing topic node as the valid position.
That is, for each virtual topic node $\vtopic{j}$ (referring to a new child of a topic node $\topic{j}$) {one at a time}, it performs topic phrase generation and clustering (Sections~\ref{subsubsec:collection} and~\ref{subsubsec:clustering}) to discover multiple novel topic nodes at the position.

\subsubsection{Novel Topic Phrase Generation}
\label{subsubsec:collection}
Given a virtual topic node $\vtopic{j}$ and each document $\doc{i}\in\docuset$, the trained model computes the topic-document similarity score and generates a topic-conditional phrase 
$\vphrase{}=[\hat{v}_{1},\ldots,\hat{v}_{T}]$ where $\hat{v}_{t}\sim P(\token{t}|\hat{v}_{<t},\vtopic{j},\doc{i})$.
Here, the generated phrase $\vphrase{}$ is less likely to belong to the virtual topic $\vtopic{j}$ if its source document $\doc{i}$ is less relevant to the virtual topic.
% According to Equation~\eqref{eq:likelihood}, the topic-conditional likelihood of a phrase is decomposed into $P(\doc{i}|\vtopic{j})$ and $P(\vphrase{}|\doc{i},\vtopic{j})$.
Thus, we utilize the topic-document similarity score as the \textit{confidence} of the generated phrase.
To collect only qualified topic phrases, we filter out non-confident phrases whose normalized topic-document similarity is smaller than a threshold, i.e., $P(\doc{i}|\vtopic{j})\approx\text{Norm}_{\doc{i}\in\docuset}(\exp(\cvec{j}^{*\top} \bm{M} \dvec{i})) < \tau$.
% In Figure~\ref{fig:framework} Right, given a virtual topic node at the valid position, \proposed topic-conditionally generates a phrase ``coffee cakes'' from an input document.
% Then, it inserts the phrase into the collection of confident phrases for the virtual node, because its topic-document similarity is larger than a threshold, i.e., $0.84 > \tau$.
In addition to the confidence-based filtering, we exclude phrases that do not appear in the corpus at all, since they are likely implausible phrases.
This has substantially reduced the \textit{hallucination} problem of a generation model.
% In addition to the confidence-based filtering, we~exclude nonsense (or implausible) phrases that do not appear in the corpus at all, which effectively avoids the \textit{hallucination} problem of a generation model.
% This filtering process effectively excludes not only topic-irrelevant phrases but also nonsense (or unseen) ones generated from topic-irrelevant documents, which helps to reduce \textit{hallucinations}.

\subsubsection{Novel Topic Phrase Clustering}
\label{subsubsec:clustering}
To identify multiple novel topics at the position of the virtual topic node $\vtopic{j}$, we perform clustering on the phrases collected for the virtual topic.
We acquire semantic features of each phrase by averaging the \glove vectors~\cite{pennington2014glove} of word tokens in the phrase, then run $k$-means clustering with the initial number of clusters $k$ manually set.
% \footnote{The $k$-means clustering is useful to discover multiple clusters of distinct semantics.}
% The details of the clustering process are provided in Section~\ref{subsec:implementation}.
Among the clusters, we selectively identify the new topics based on their cluster size, and the center phrase of each cluster is used as the topic name.
% Based on the clustering results, we can identify the new topic nodes, where the center phrase of each cluster is used as the final topic name.
% In Figure~\ref{fig:framework} Right, \proposed successfully finds out two novel topics, ``cakes'' and ``cookies'', along with their topic-related phrases, by clustering the collected phrases for the target virtual node.
% Finally, it inserts the new topic nodes into the target position, which is the child (i.e, subtopic) of ``breads bakery''.



\section{Experiments}
\label{sec:exp}
% In this section, we present the experimental results that validate the superiority of \proposed.

\subsection{Experimental Settings}
\label{subsec:expsetting}
\smallsection{Datasets}
\label{subsubsec:dataset}
% For our experiments,
We use two real-world document corpora with their three-level topic taxonomy: \textbf{\amazon}~\cite{mcauley2013hidden} contains product reviews collected from Amazon, and \textbf{\dbpedia}~\cite{lehmann2015dbpedia} contains Wikipedia articles.
All the documents in both datasets are tokenized by the BERT tokenizer~\cite{devlin2019bert} and truncated to have maximum 512 tokens.
The statistics are listed in Table~\ref{tbl:datastats}.

\smallsection{Baseline Methods}
\label{subsubsec:baseline}
We consider methods for building a topic taxonomy from scratch, \textbf{\hlda} \cite{blei2003hierarchical} and \textbf{\taxogen} \cite{zhang2018taxogen}. 
We also evaluate the state-of-the-art methods for topic taxonomy expansion,
\textbf{\corel} \cite{huang2020corel} and \textbf{\taxocom} \cite{lee2022taxocom}.\footnote{The implementation details and hyperparameter selection for the baselines and \proposed are in Sections~\ref{subsec:basedetail} and~\ref{subsec:implementation}.}
Both of them identify and insert new topic nodes based on term embedding and clustering, with the initial topic taxonomy leveraged as supervision.
% We employ the official author codes of \corel\footnote{https://github.com/teapot123/CoRel} and \taxocom\footnote{https://github.com/donalee/taxocom} for experiments.

%\footnote{https://github.com/joewandy/hlda}
%\footnote{https://github.com/franticnerd/taxogen}
%\footnote{https://github.com/teapot123/CoRel}
%\footnote{https://github.com/donalee/taxocom}

% \begin{itemize}
%     \item \textbf{\hlda}~\cite{blei2003hierarchical}: Hierarchical latent Dirichlet allocation for probabilistic topic modeling.
    
%     \item \textbf{\taxogen}~\cite{zhang2018taxogen}: The recursive framework based on text embedding and clustering.
    
%     \item \textbf{\corel}~\cite{huang2020corel}: 
%     The first topic taxonomy expansion method.
%     It trains a topic relation classifier by using the initial taxonomy, then recursively transfers the relation to find out candidate terms for novel subtopics. 
%     Finally, it identifies novel topic nodes based on term embeddings induced by SkipGram~\cite{mikolov2013distributed}.
    
%     \item \textbf{\taxocom}~\cite{lee2022taxocom}: 
%     The state-of-the-art method for topic taxonomy completion. 
%     For each node from the root to the leaf, it recursively optimizes term embedding and performs term clustering to identify both known and novel subtopics.
    
%     \item \textbf{\proposed}: The proposed framework for topic taxonomy expansion.
%     It trains a topic-conditional phrase generator by using the initial topic taxonomy, then generates topic-related phrases for a virtual topic node whose semantics is captured based on its surrounding relation structure in the hierarchy.
% \end{itemize}

\smallsection{Experimental Settings}
To evaluate the performance for novel topic discovery, 
%we run each expansion method using an incomplete topic taxonomy which is made by randomly deleting half of leaf topic nodes from the original topic taxonomy, as done in~\cite{lee2022taxocom}.
we follow the previous convention that randomly deletes half of leaf nodes from the original taxonomy and asks each expansion method to reproduce them~\cite{shen2020taxoexpan, lee2022taxocom}.
%\footnote{In detail, 259 topic nodes (for \amazon) and 109 topic nodes (for \dbpedia) are deleted.} 
Considering the deleted topics as \textit{ground-truth}, we measure how completely new topics are discovered and how accurately they are inserted into the taxonomy.
% Three different parts of the topic taxonomy $\subtaxo{1}$, $\subtaxo{2}$, and $\subtaxo{3}$ are given as the initial taxonomy for the expansion methods, each of which covers some of the first-level topics (and their subtrees) listed in Table~\ref{tbl:taxopart}.

% \subsection{Experimental Results}
% \label{subsec:expresult}

\begin{table}[t]
\small
\caption{The statistics of the datasets.}
\label{tbl:datastats}
\centering
\resizebox{0.99\linewidth}{!}{%
\begin{tabular}{cccc}
    \toprule
        \textbf{Corpus} & \textbf{Vocab. size} & \textbf{\# Documents} & \textbf{\# Topic nodes} \\\midrule
        \amazon & 19,615 & \ \ 29,487 & 531 \\
        \dbpedia & 27,435 & 196,665 & 298 \\
    \bottomrule
\end{tabular}
}
\end{table}


\begin{table*}[t]
\caption{Quantitative evaluation on output topic taxonomies. 
%The best results are marked in boldface.
% The evaluation results obtain the Kendall coefficient of 0.96/0.91/0.84 (\amazon) and 0.93/0.90/0.91 (\dbpedia) respectively for each aspect, which indicates strong inter-rater agreement on ranks of the methods.
The average and standard deviation for the three aspects are reported. 
The relation accuracy and subtopic integrity are considered only for the expansion methods, whose identified new topic nodes can be clearly compared with the ground-truth ones at each valid position.
}
\centering
\resizebox{0.99\linewidth}{!}{%
\begin{tabular}{ccPPPPPP}
    \toprule
     \multirow{2.5}{*}{\shortstack{\textbf{Part}}} & \multirow{2.5}{*}{\shortstack{\textbf{Methods}}} & \multicolumn{3}{c}{\textbf{\amazon}} & \multicolumn{3}{c}{\textbf{\dbpedia}} \\\cmidrule(lr){3-5}\cmidrule(lr){6-8}
    & & {{\small {\textbf{Term Coherence}}}} & {{\small {\textbf{Relation Accuracy}}}} & {{\small {\textbf{Subtopic Integrity}}}} & {{\small \textbf{Term Coherence}}} & {{\small \textbf{Relation Accuracy}}} & {{\small \textbf{Subtopic Integrity}}} \\\midrule
    
    \multirow{2}{*}{}
    & \hlda & 0.2417  (0.0398) &  N/A & N/A & 0.2688  (0.0320) &  N/A &  N/A \\
    & \taxogen & 0.4333  (0.1062) &  N/A &  N/A & 0.4906  (0.1523) &  N/A & N/A \\\midrule
    
    \multirow{3}{*}{$\subtaxo{1}$}
    & \corel & 0.5167  (0.1512) & 0.4833  (0.1501) & 0.2708  (0.1263) & 0.5083  (0.1377) & 0.6583  (0.1762) & 0.2813  (0.1727) \\
    & \taxocom & 0.6667  (0.1411) & 0.5167  (0.0992) & 0.3177  (0.1006) & 0.5250  (0.2151) & 0.6833  (0.1808) & 0.2917  (0.1282) \\
    \rowcolor{Gray} \cellcolor{white} 
    & \proposed & \textbf{0.9750}  (0.0496) & \textbf{0.8833}  (0.1113) & \textbf{0.4948}  (0.1309) & \textbf{0.9667}  (0.0713) & \textbf{0.9333}  (0.0713) & \textbf{0.5781}  (0.1389)  \\\midrule
    
    \multirow{3}{*}{$\subtaxo{2}$}
    & \corel & 0.5583  (0.1967) & 0.6333  (0.1594) & 0.2569  (0.1215) & 0.4417  (0.1815) & 0.5583  (0.1231) & 0.1458  (0.1488) \\
    & \taxocom & 0.6083  (0.1466) & 0.6167  (0.1369) & 0.4514  (0.1464) & 0.4833  (0.1944) & 0.7083  (0.1467) & 0.2708  (0.1282)\\
    \rowcolor{Gray} \cellcolor{white} 
    & \proposed & \textbf{0.8917}  (0.0707) & \textbf{0.8583}  (0.1650) & \textbf{0.6597}  (0.2062) & \textbf{0.9583}  (0.0707) & \textbf{0.9167}  (0.1054) & \textbf{0.5729}  (0.1035) \\\midrule
    
    \multirow{3}{*}{$\subtaxo{3}$} 
    & \corel & 0.5667  (0.1638) & 0.5833  (0.1222) & 0.2344  (0.1527) & 0.6250  (0.1669) & 0.7167  (0.1321) & 0.3177  (0.1195) \\
    & \taxocom & 0.5917  (0.1571) & 0.6083  (0.0972) & 0.1563  (0.1179) & 0.5667  (0.1852) & 0.6917  (0.1179) & 0.4167  (0.1069) \\
    \rowcolor{Gray} \cellcolor{white} 
    & \proposed & \textbf{0.9167}  (0.0690) & \textbf{0.9083}  (0.1004) & \textbf{0.4531}  (0.1068) & \textbf{0.9833}  (0.0309) & \textbf{0.9417}  (0.0904) & \textbf{0.6719}  (0.0916) \\\bottomrule
\end{tabular}
}
\label{tbl:humaneval}
\end{table*}

\begin{table}[t]
\caption{Performance for topic phrase generation.}
\centering
\resizebox{0.99\linewidth}{!}{%
\begin{tabular}{lrrrr}
    \toprule
    \multirow{2.5}{*}{\textbf{Methods}} & \multicolumn{2}{c}{\textbf{\amazon}} & \multicolumn{2}{c}{\textbf{\dbpedia}} \\\cmidrule(lr){2-3}\cmidrule(lr){4-5}
    & \textbf{PPL} $\downarrow$ & \textbf{ACC} $\uparrow$ & \textbf{PPL} $\downarrow$ & \textbf{ACC} $\uparrow$ \\\midrule
    \rowcolor{Gray}
    \proposed & \textbf{5.2553} & \textbf{0.6958}  & \textbf{3.1108} & \textbf{0.7768} \\
    {\small \ \ (Encoder) BERT$\rightarrow$Bi-GRU} & 5.7844 & 0.6884 & 3.5322 & 0.7645 \\
    {\small \ \ (Decoder) Transformer$\rightarrow$GRU} & 6.6649 & 0.6754 & 5.3690 & 0.6798 \\
    {\small \ \ w/o Topic-attentive context} & 7.0907 & 0.6643 & 7.1679 & 0.6553\\
    {\small \ \ w/o Hierarchical topic relation} & 6.5345 & 0.6772 & 3.9802 & 0.7423 \\
    {\small \ \ w/o Sideward topic relation} & 5.8705 & 0.6807 & 3.6985 & 0.7506 \\
    \midrule
    
    \multicolumn{2}{l}{{\footnotesize \textrank~\cite{mihalcea2004textrank}}\hfill -} & 0.3023 & \multicolumn{1}{c}{$~~~$-} & 0.1628 \\
    \multicolumn{2}{l}{{\footnotesize \topicrank~\cite{bougouin2013topicrank}}\hfill $~~~~$-} & 0.2099 & \multicolumn{1}{c}{$~~~$-} & 0.1092 \\
    {\small \topickg~\cite{wang2019topic}} & 13.1298 & 0.2770 & 11.5663 & 0.3238 \\
    {\small \bertkg~\cite{liu2020keyphrase}} & 11.0229 & 0.4165 & 9.4723 & 0.4734 \\
    {\small \berttkg~\cite{zhou2021topic}} & 10.9746 & 0.4308 & 8.3607 & 0.4894 \\
    \bottomrule
\end{tabular}
}
\label{tbl:genperf}
\end{table}


\subsection{Quantitative Evaluation}
\subsubsection{Topic Taxonomy Expansion}
\label{subsubsec:humaneval1}
First of all, we assess the quality of output topic taxonomies.
Following previous topic taxonomy evaluations~\cite{huang2020corel, lee2022taxocom}, we recruit 10 doctoral researchers and use their domain knowledge to examine three different aspects of a topic taxonomy.
%\footnote{The scores are reported after being averaged over all evaluators and all target nodes.}
\textbf{Term coherence} indicates how strongly terms in a topic node are relevant to each other. 
% Human evaluators count the number of terms that are relevant to the common topic (or topic name) among the top-5 terms found for each topic node.
\textbf{Relation accuracy} computes how accurately a topic node is inserted into the topic taxonomy (i.e., \textit{precision} for novel topic discovery).
% For each valid position, human evaluators count the number of newly-inserted topics that are in the correct relationship with the surrounding topics.
\textbf{Subtopic integrity} measures the completeness of subtopics for a topic node (i.e., \textit{recall} for novel topic discovery).
% Human evaluators investigate how many ground-truth novel topics, which were deleted from the original taxonomy (Section~\ref{subsubsec:dataset}), match with one of the newly-inserted topics.
% For practical evaluation on a large-scale topic taxonomy with hundreds of topic nodes,
% We divide each output taxonomy into three parts $\subtaxo{1}$, $\subtaxo{2}$, and $\subtaxo{3}$ so that each of them covers some of the first-level topics (and their subtrees) listed in Table~\ref{tbl:taxopart}.
For exhaustive evaluation, we divide the output taxonomy of each expansion method into three disjoint parts $\subtaxo{1}$, $\subtaxo{2}$, and $\subtaxo{3}$ so that each of them covers some first-level topics (and their subtrees) in Table~\ref{tbl:taxopart} in Section~\ref{subsec:evalprotocol}.\footnote{The details of the evaluation protocol are in Section~\ref{subsec:evalprotocol}.}

In Table~\ref{tbl:humaneval}, \proposed achieves the highest scores for all the aspects.\footnote{The evaluation results obtain the Kendall coefficient of 0.96/0.91/0.84 (\amazon) and 0.93/0.90/0.91 (\dbpedia) respectively for each aspect, which indicates strong inter-rater agreement on ranks of the methods.}
For all the baseline methods, the term coherence is not good enough~because they assign candidate terms into a new topic according to the topic-term relevance mostly learned~from term co-occurrences.
In contrast, \proposed effectively collects coherent terms relevant to a new topic (i.e., term coherence $\geq 0.90$) by directly generating the topic-conditional terms from documents.
\proposed also shows significantly higher relation accuracy and subtopic integrity than the other expansion methods, with the help of its GNN-based topic encoder that captures a holistic topic structure beyond the first-order topic relation.
% Most of its topic phrases in each topic node are semantically relevant, i.e., term coherence $\geq 0.90$, and also most of its inserted new topics do not break relation consistency of the topic hierarchy, i.e., relation accuracy $\geq 0.85$.
% \footnote{Note that only a few terms (and new topics) are presented to evaluators for each topic node (and each valid position), so the reported values of standard deviation smaller than 0.2 actually imply consistent assessment across the evaluators.}
% \proposed also shows significantly better subtopic integrity than the other baseline methods.
% We remark that the existing expansion methods have been validated only for a small topic hierarchy with tens of topic nodes~\cite{huang2020corel, lee2022taxocom};
% that is, our results indicate that they do not work well for a larger and deeper topic hierarchy.
% In conclusion, our framework can obtain a better expanded topic taxonomy than all the baselines, with the help of hierarchy-aware topic phrase generation.



% \begin{table}[t]
% \caption{Topic terms retrieved by each method. Irrelevant terms are colored in red, and multi-word terms that are not in the candidate term set of \autophrase are in boldface.}
% \centering
% \resizebox{0.99\linewidth}{!}{%
% \begin{tabular}{cl}
%     \toprule
    
%     \multicolumn{2}{l}{\textbf{\amazon} (Root $\rightarrow$ grocery gourmet food $\rightarrow$ beverages $\rightarrow$ tea)} \\\midrule
%     \corel & blend, alvita, green tea, white tea, herbal tea \\
%     \taxocom & earl grey, taylors of harrogate, black tea, herbal tea, teabags \\
%     {\small \proposed} & Green tea, herbal tea, black tea, vanilla chai tea, organic teas \\\midrule
    
%     \multicolumn{2}{l}{\textbf{\amazon} (Root $\rightarrow$ health personal care $\rightarrow$ household supplies $\rightarrow$ dishwashing)} \\\midrule
%     \corel & household, fresh, flushes, dirty, biological \\
%     \taxocom & dishwasher, bubbles, dirty, smelly, nasty \\
%     {\small \proposed} & dishwasher, dish soap, \textbf{dish detergent}, \textbf{dishwasher gel}, detergent \\\midrule
    
%     \multicolumn{2}{l}{\textbf{\amazon} (Root $\rightarrow$ pet supplies $\rightarrow$ fish aquatic pets $\rightarrow$ breeding tanks)} \\\midrule
%     \corel & aquarium, fish, tank, litter, litter box \\
%     \taxocom & goldfish, aquarium, fish, acrylic, tank \\
%     {\small \proposed} &  \\\midrule

%     \multicolumn{2}{l}{\textbf{\dbpedia} (Root $\rightarrow$ work $\rightarrow$ periodical literature $\rightarrow$ magazine)} \\ \midrule
%     \corel & science fiction, comics, anthology, monthly, newspaper \\
%     \taxocom & books, cover, fashion, specializing, publishers \\
%     \multirow{2}{*}{\small \proposed} & weekly magazine, news magazine, \textbf{irish magazine} \\
%     & \textbf{fantasy comics magazine}, \textbf{international reportage} \\\midrule
    
%     \multicolumn{2}{l}{\textbf{\dbpedia} (Root $\rightarrow$ event $\rightarrow$ societal event $\rightarrow$ film festival)
%     } \\ \midrule
%     \corel & international film festival, music festival, annual event, festival, cinema \\
%     \taxocom & international film festival, fest, festival, independent film, annual festival \\
%     \multirow{2}{*}{\small \proposed} & international film festival, \textbf{melbourne international film festival}, \\
%     & annual film festival, \textbf{cannes film festival}, \textbf{venice film festival} \\\midrule
    
%     \multicolumn{2}{l}{\textbf{\dbpedia} (Root $\rightarrow$ sports season $\rightarrow$ sports team season $\rightarrow$ ncaa team season)} \\ \midrule
%     \corel & naia, division ii, ncaa division ii, atlantic coast conference, mid american conferencer \\
%     \taxocom & mastodons, basketball tournament, michigan state university, bulldogs men, lehigh mountain hawks \\
%     \multirow{2}{*}{\small \proposed} & ncaa national team, ncaa tournament appearancel ncaa division iii, \\
%     & ncaa division ii men, ncaa championship golf team \\
%     \bottomrule
% \end{tabular}
% }
% \label{tbl:humaneval}
% \end{table}

\begin{table*}[t]
\caption{Top-5 topic terms included in each topic node. The off-topic (or too general) terms are marked with a strikethrough, and the multi-word terms that are not obtainable by the extraction-based methods are underlined.}
\centering
\setlength{\tabcolsep}{4pt}
\resizebox{\linewidth}{!}{%
\begin{tabular}{cQR}
    \toprule
    & \multicolumn{1}{l}{\textbf{\amazon}} & \multicolumn{1}{l}{\textbf{\dbpedia}} \\\midrule
    
    \textbf{Topic}
    & \reduce{\textbf{Root $\rightarrow$ grocery gourmet food $\rightarrow$ beverages $\rightarrow$ tea}}
    & \reduce{\textbf{Root $\rightarrow$ work $\rightarrow$ periodical literature $\rightarrow$ magazine}} \\\cmidrule(lr){1-1}\cmidrule(lr){2-2}\cmidrule(lr){3-3}
    
    \reduce{\corel}
    & \reduce{blend, alvita, green tea, white tea, herbal tea}
    & \reduce{science fiction, comics, anthology, \wrong{monthly}, newspaper} \\
    
    \reduce{\taxocom}
    & \reduce{earl grey, taylors of harrogate, black tea, herbal tea, teabags} 
    & \reduce{\wrong{books}, cover, fashion, \wrong{specializing}, publishers} \\
    
    \reduce{\proposed}
    & \reduce{green tea, herbal tea, black tea, \unique{vanilla chai tea}, \unique{jasmine tea}}
    & \reduce{news mag.., monthly mag.., \unique{business mag..}, \unique{comics mag..}, \unique{fashion mag..}} \\\midrule
    
    \textbf{Topic}
    & \reduce{\textbf{Root $\rightarrow$ personal care $\rightarrow$ household supplies $\rightarrow$ dishwashing}}
    & \reduce{\textbf{Root $\rightarrow$ event $\rightarrow$ societal event $\rightarrow$ film festival}} \\\cmidrule(lr){1-1}\cmidrule(lr){2-2}\cmidrule(lr){3-3}
    
    \reduce{\corel} 
    & \reduce{\wrong{household}, fresh, flushes, dirty, \wrong{biological}} 
    & \reduce{film festival, \wrong{music festival}, annual event, \wrong{festival}, cinema} \\
    
    \reduce{\taxocom}
    & \reduce{dishwasher, bubbles, dirty, smelly, nasty}
    & \reduce{film festival, \wrong{fest}, \wrong{festival}, independent film, annual festival} \\
    
    \reduce{\proposed}
    & \reduce{dishwasher, dish soap, \unique{dish detergent}, \unique{dishwasher gel}, \unique{dishwasher soap}} 
    & \reduce{film festi.., \unique{short film festi..}, annual film festi.., \unique{cannes film festi..}, \unique{venice film festi..}}\\\midrule
    
    \textbf{Topic}
    & \reduce{\textbf{Root $\rightarrow$ baby products $\rightarrow$ diapering $\rightarrow$ cloth diapers}}
    & \reduce{\textbf{Root $\rightarrow$ sports season $\rightarrow$ sports team season $\rightarrow$ ncaa team season}} \\\cmidrule(lr){1-1}\cmidrule(lr){2-2}\cmidrule(lr){3-3}
    
    \reduce{\corel}
    & \reduce{cloth diaper, \wrong{diapers}, \wrong{diaperbag}, cloth diapering, {baby wipes}}
    & \reduce{naia, div.. ii, ncaa div.. ii, atlantic coast conference, mid american conference} \\
    
    \reduce{\taxocom}
    & \reduce{\wrong{disposable diapers}, \wrong{biodegradeable}, cloth diaper, \wrong{diapering}, bulky}
    & \reduce{mastodons, basketball tournament, college football, bulldogs, hoosiers} \\
    
    \reduce{\proposed}
    & \reduce{cloth diaper, prefold diaper, \unique{diaper cover}, \unique{diaper liner}, \unique{reusable diaper}}
    & \reduce{\unique{ncaa national team}, ncaa tournament, ncaa div.. ii, ncaa div.. i, \unique{ncaa championship}} \\

    \bottomrule
\end{tabular}
}
\label{tbl:topicterms}
\end{table*}


% \begin{table}[t]
% \caption{Novel topics newly-inserted at the target position. Redundant topics (\ding{34}) and incorrect topics ($\otimes$) are additionally annotated. Only the center terms (i.e., topic names) are presented in alphabetical order.}
% \centering
% \resizebox{\linewidth}{!}{%
% \begin{tabular}{ccS}
%     \toprule
%     \multirow{10}{*}{\rotatebox[origin=c]{90}{\textbf{\amazon}}}
%     & {\textbf{Position}}
%     & \reduce{\textbf{Root $\rightarrow$ grocery gourmet food $\rightarrow$ beverages $\rightarrow$ ?}}
%     \\\cmidrule(lr){2-2}\cmidrule(lr){3-3}
    
%     & {\textbf{Sibling topics}}
%     & \reduce{\textbf{tea, coffee, hot cocoa, water, sports drinks}}
%     \\\cmidrule(lr){2-2}\cmidrule(lr){3-3}
    
%     & \corel 
%     & \reduce{apple cider, bottles ($\otimes$), drinking ($\otimes$), fruit juice, matcha (\ding{34})} \\
    
%     & \taxocom 
%     & \reduce{decaf chai (\ding{34}), espresso (\ding{34}), fizzy, juice, lipton (\ding{34}), mouthwash ($\otimes$)} \\
    
%     & \proposedab
%     & \reduce{breakfast tea (\ding{34}), coconut water, fruit juice, natural cocoa (\ding{34}), vanilla coffee (\ding{34})} \\
    
%     & \proposedfull
%     & \reduce{coconut water, cream soda, decaf tea (\ding{34}), diet smoothie, redline energy drink} \\\midrule
    
%     \multirow{10.5}{*}{\rotatebox[origin=c]{90}{\textbf{\dbpedia}}}
%     & {\textbf{Position}}
%     & \reduce{\textbf{Root $\rightarrow$ agent $\rightarrow$ sports team $\rightarrow$ ?}}
%     \\\cmidrule(lr){2-2}\cmidrule(lr){3-3}
    
%     & {\textbf{Sibling topics}}
%     & \reduce{\textbf{basketball team, cycling team, football team}} \\\cmidrule(lr){2-2}\cmidrule(lr){3-3}
    
%     & \corel 
%     & \reduce{baseball ($\otimes$), domestic competition ($\otimes$), football club (\ding{34}), national ice hockey team, soccer ($\otimes$)} \\
    
%     & \taxocom 
%     & \reduce{hockey ($\otimes$), junior football team (\ding{34}), national team, regular season ($\otimes$), rugby club} \\
    
%     & \proposedab
%     & \reduce{american football team (\ding{34}), cricket team, cycling team (\ding{34}), professional basketball team (\ding{34}), rugby union team} \\
    
%     & \proposedfull
%     & \reduce{beach handball team, cricket team, football club (\ding{34}), ice hockey team, rugby union team} \\
%     \bottomrule
% \end{tabular}
% }
% \label{tbl:noveltopics}
% \end{table}


\begin{table*}[t]
\caption{Novel topics identified at each target position. The center term (i.e., topic name) of each identified topic is presented. Correct topics ($\Circle$), incorrect topics ($\otimes$), and redundant topics (\ding{34}) are annotated.}
\setlength{\tabcolsep}{4pt}
\centering
\resizebox{\linewidth}{!}{%
\begin{tabular}{cSW}
    \toprule
    & {\textbf{\amazon}} &  {\textbf{\dbpedia}} \\ \midrule
    
    \reduce{\textbf{Position}}
    & \reduce{\textbf{Root $\rightarrow$ grocery gourmet food $\rightarrow$ beverages $\rightarrow$ ?}} 
    & \reduce{\textbf{Root $\rightarrow$ agent $\rightarrow$ sports team $\rightarrow$ ?}}
    \\\cmidrule(lr){1-1}\cmidrule(lr){2-2}\cmidrule(lr){3-3}
    
    \reduce{\textbf{Sibling topics}}
    & \reduce{\textbf{tea, coffee, hot cocoa, water, sports drinks}} 
    & \reduce{\textbf{basketball team, cycling team, football team}}
    \\\cmidrule(lr){1-1}\cmidrule(lr){2-2}\cmidrule(lr){3-3}
    
    \reduce{\corel} 
    & \reduce{apple cider ($\Circle$), bottles ($\otimes$), drinking ($\otimes$), fruit juice ($\Circle$), matcha (\ding{34})} 
    & \reduce{baseball ($\otimes$), domestic competition ($\otimes$), football club (\ding{34}), national~ice~hockey~team ($\Circle$), soccer ($\otimes$)} \\
    
    \reduce{\taxocom} 
    & \reduce{decaf chai (\ding{34}), espresso (\ding{34}), fizzy ($\Circle$), juice ($\Circle$), lipton~(\ding{34}), mouthwash ($\otimes$)} 
    & \reduce{hockey ($\otimes$), junior football team (\ding{34}), national team ($\Circle$), regular~season~($\otimes$), rugby~club ($\Circle$)} \\
    
    \reduce{\proposedab}
    & \reduce{breakfast tea (\ding{34}), coconut water ($\Circle$), fruit juice ($\Circle$), natural~cocoa (\ding{34}), vanilla coffee (\ding{34})} 
    & \reduce{american~football~team (\ding{34}), cricket~team ($\Circle$), cycling~team (\ding{34}), professional~basketball~team (\ding{34}), rugby~union~team ($\Circle$)} \\
    
    \reduce{\proposedfull}
    & \reduce{coconut water ($\Circle$), cream soda ($\Circle$), decaf tea (\ding{34}), diet~smoothie ($\Circle$), redline energy drink ($\Circle$)} 
    & \reduce{beach~handball~team ($\Circle$), cricket~team ($\Circle$), football club (\ding{34}), ice~hockey~team ($\Circle$), rugby~union~team ($\Circle$)} \\
    \bottomrule
\end{tabular}
}
\label{tbl:noveltopics}
\end{table*}


\subsubsection{Topic-Conditional Phrase Generation}
\label{subsubsec:ablation}
We investigate the topic phrase prediction performance of our framework and other keyphrase extraction/generation models.
% To validate the effectiveness of each component in our framework, we investigate the phrase generation performance for an ablation study:
% 1) BERT encoder (vs. bidirectional-GRU encoder), 
% 2) transformer decoder (vs. GRU decoder), 
% 3) topic-attentive context representation (vs. simply concatenating the representation of a topic and that of each contextualized token), and
% 4) GCN topic encoder with sideward relation modeling (vs. without sideward relation modeling).
% 5) GCN topic encoder with hierarchical relation modeling (vs. without hierarchy-awareness).      
We leave out 10\% of the positive triples $(\topic{j}, \doc{i}, \phrase{k})$ from the training set $\mathcal{X}$ and use them as the test set.
We measure \textbf{perplexity (PPL)} and \textbf{accuracy (ACC)} by comparing each generated phrase with the target phrase at the token-level and phrase-level, respectively.
% \begin{equation*}
% \small
% \begin{split}
% \textstyle
%     \text{PPL} &= \frac{1}{|\mathcal{X}_{test}|} \sum_{(\topic{j}, \doc{i}, \phrase{k})\in\mathcal{X}_{test}}\sum _{t=1}^T \ -\log P(\token{kt}|\token{k(<t)},\topic{j}, \doc{i}) \\
%     \text{ACC} &=  \frac{1}{|\mathcal{X}_{test}|} \sum_{(\topic{j}, \doc{i}, \phrase{k})\in\mathcal{X}_{test}} \prod_{t=1}^T \mathbb{I}[\hat{v}_{kt} == \token{kt}]
% \end{split}
% \end{equation*}
% We repeat the experiment three times and report their average.

In Table~\ref{tbl:genperf}, \proposed achieves the best PPL and ACC scores.
We observe that \proposed more accurately generates topic-related phrases from input documents, compared to the state-of-the-art keyphrase generation methods which are not able to consider a specific topic as the condition for generation.
% We observe that utilizing the advanced neural architecture to encode and decode texts (i.e., transformer) as well as leveraging the pretrained checkpoint (i.e., BERT) contribute to accurate generation of topic phrases.
In addition, ablation analyses validate that each component of our framework contributes to accurate generation of topic phrases. 
% topic-attentive context representations play a critical role in precisely specifying the condition for generating a topic phrase from an input document.
Particularly, the hierarchical (i.e., upward and downward) and sideward relation modeling of the topic encoder improves the quality of generated phrases.
%by encouraging a topic representation to encode distinct semantics of a relation structure.


\subsection{Qualitative Evaluation}
\subsubsection{Comparison of Topic Terms}
\label{subsubsec:topicterms}
We qualitatively compare the topic terms found by each method.
%, we list the top-5 terms of three topic nodes for each dataset.
% Note that the extraction-based methods (i.e., \corel and \taxocom) retrieve topic terms from the predefined set of candidate terms (extracted by \autophrase~\cite{shang2018automated}) by using their relevance score to each topic, whereas the generation-based method (i.e., \proposed) sequentially decodes word tokens given a topic and an input document.
In case of \proposed, we sort all confident topic terms by their cosine distances to the topic name (i.e., center term) using the global embedding features~\cite{pennington2014glove}.
%, as done in Section~\ref{subsubsec:expansion}.

Table~\ref{tbl:topicterms} shows that the topic terms of \proposed are superior to those of the baseline methods, in terms of the expressiveness as well as the topic relevance.
In detail, some of the terms retrieved by \corel and \taxocom are either off-topic or too general (marked with a strikethrough);
this indicates that their topic relevance score for each term is not good at capturing the hierarchical topic knowledge of a text corpus.
% On the contrary, \proposed generates strongly topic-related terms from relevant documents, by using the relation structure of a target topic as the condition for generation.
On the contrary, \proposed generates strongly topic-related terms by capturing the relation structure of each topic.
Furthermore, \proposed is effective to find infrequently-appearing multi-word terms (underlined), which all the extraction-based methods fail to obtain.

\begin{figure}[t]
    \centering
    \includegraphics[width=\linewidth]{FIG/casestudy.pdf}
    \caption{Examples of topic-conditional phrase generation, given a document and its relevant/irrelevant topic.}
    \label{fig:casestudy}
\end{figure}

\subsubsection{Comparison of Novel Topics}
\label{subsubsec:noveltopics}
Next, we examine novel topics inserted by each expansion method.
To show the effectiveness of sideward relation modeling adopted by our topic encoder (Section~\ref{subsubsec:topic_encoder}),
we additionally present the results of \textbf{\proposedfull} and \textbf{\proposedab}, which computes topic representations with and without capturing the \underline{s}ideward topic \underline{r}elations.

In Table~\ref{tbl:noveltopics}, \proposedfull successfully discovers new topics that should be placed in a target position.
Notably, the new topics are clearly distinguishable from the sibling topics (i.e., known topics given in the initial topic hierarchy), which reduces the redundancy of the output topic taxonomy.
On the other hand, \corel and \taxocom show limited performance for novel topic discovery; 
some new topics are redundant (\ding{34}) while some others do not preserve the hierarchical relation with the existing topics ($\otimes$).
Some of the new topics found by \proposedab semantically overlap with the sibling topics, even though they are at the correct position in the hierarchy;
this implies that our topic encoder with sideward relation modeling makes the representation of a virtual topic node \textit{discriminative} with its sibling topic nodes, and it eventually helps to discover new conceptual topics of novel semantics.

\subsubsection{Case Study of Topic Phrase Generation}
\label{subsubsec:casestudy}
To study how the generated phrases and their topic-document similarity scores (i.e., confidences) vary depending on a topic condition, we provide examples of topic-conditional phrase generation.
The input document in Figure~\ref{fig:casestudy} contains a review about nail care products. 
In case that the relation structure of a target topic implies the nail product (Figure~\ref{fig:casestudy} Left),
%(i.e., the position of a node annotated with \texttt{[MASK]} in Figure~\ref{fig:casestudy} Left), 
\proposed obtains the desired \textit{topic-relevant} phrase ``nail lacquer'' along with the high topic-document similarity of 0.8547.
On the other hand, given the relation structure of a target topic which is inferred as a kind of meat foods (Figure~\ref{fig:casestudy} Right),
%(i.e., the position of a node annotated with \texttt{[MASK]} in Figure~\ref{fig:casestudy} Right),
it generates a \textit{topic-irrelevant} phrase ``metallic black'' from the document along with the low topic-document similarity of 0.0023.
That is, \proposed fails to get a qualified topic phrase when the textual contents of an input document is obviously irrelevant to a target topic. 
In this sense, \proposed filters out non-confident phrases having a low topic-document similarity score to collect only the phrases relevant to each virtual topic.
% Please refer to the supplementary material for more examples of topic-conditional phrase generation.

\subsection{Analysis of Topic-Document Similarity}
\label{subsec:unseenphs}
Finally, we investigate the changes of generated phrases in two aspects, with respect to the topic-document similarity scores.
The first aspect is the ratio of three categories for generated phrases, which have been focused on in the literature of keyphrase generation~\cite{meng2017deep, zhou2021topic}: 
\textbf{(1) present phrases} appearing in the input document, 
\textbf{(2) absent phrases} not appearing in the input document but in the corpus at least once, 
and \textbf{(3) unseen (i.e., totally-new) phrases} that are not observed in the corpus at all.
The second aspect is the average semantic distance among the phrases, measured by using the semantic features. %~\cite{pennington2014glove}.
For the plots in Figure~\ref{fig:confanal}, the horizontal axis represents 10 bins of normalized topic-document similarity scores over all generated phrases.

\begin{figure}[t]
    \centering
    \includegraphics[width=\linewidth]{FIG/confanal.pdf}
    \caption{\reducetxt{The ratio of three categories for generated phrases (Left) and the average semantic distance among generated phrases (Right). The horizontal axis shows 10 bins of normalized topic-document similarity scores.}}
    \label{fig:confanal}
\end{figure}

Interestingly, \proposed hardly generates absent phrases (about $0.7\%$ for \amazon, $1.7\%$ for \dbpedia) and unseen phrases (about $0.1\%$ for \amazon, $0.2\%$ for \dbpedia) regardless of the topic-document similarity; instead, it generates present phrases in most cases (Figure~\ref{fig:confanal} Left). 
In other words, if the input document is not relevant to a target topic, it tends to generate an irrelevant-but-present phrase rather than a relevant-but-absent phrase, as shown in Section~\ref{subsubsec:casestudy}. 
One potential risk of \proposed is to generate unseen phrases that are nonsense or implausible, also known as \textit{hallucinations} in neural text generation, and such unseen phrases can degrade the quality and credibility of output topic taxonomies. 
This result supports that we can easily exclude all unseen phrases, which account for less than 0.2\% of generated phrases, to effectively address this issue.

Moreover, the negative correlation between the topic-document similarity score and the inter-phrase semantic distance (Figure~\ref{fig:confanal} Right) provides empirical evidence that the similarity score can serve as the confidence of a generated topic phrase.
There is a clear tendency toward decreasing the average semantic distance as the topic-document similarity score increases;
this implies that the phrases generated from topic-relevant documents are semantically coherent to each other, and accordingly, they are likely to belong to the same topic.



\section{Conclusion}
\label{sec:conc}
In this paper, we study the problem of topic taxonomy expansion, pointing out that the existing approach has shown limited term coverage and inconsistent topic relation.
Our \proposed framework introduces hierarchy-aware topic term generation, which generates a topic-related term by using both the textual content of an input document and the relation structure of a topic as the condition for generation.
% In brief, the training step optimizes the unified neural model to maximize the total likelihood of the initial topic taxonomy given a text corpus, then the expansion step discovers new topics at each valid position in the topic hierarchy by collecting confident phrases generated by the trained model.
The quantitative and qualitative evaluation demonstrates that our framework successfully obtains much higher-quality topic taxonomy in various aspects, compared to other baseline methods.  

For future work, it would be promising to incorporate an effective measure for the topic relevance of multi-word terms (i.e., phrases) into our framework.
%, for the sake of accurate consideration of the semantic relevance between each topic and topic-related terms.
Since learning and utilizing the representation of multi-word terms remains challenging and worth exploring, it can be widely applied to many other text mining tasks.


\section{Limitations}
\label{sec:limit}
Despite the remarkable performance of \proposed on our tested corpus, there is still room to improve regarding how to better handle topics, documents, and phrases, for effective mining of topic knowledge.
First, \proposed uses only the topic names (i.e., center terms) as the base node features in the topic relation graph, which makes our topic encoder difficult to capture the collective meaning of each topic from its set of topic-related phrases.
Second, the confidence of each generated phrase considers only the topic relevance of its source document, instead of all the documents in which this phrase appears.
Finally, the clustering process does not leverage the contextualized textual features computed by our BERT-based document encoder, which makes it hard to consolidate the context of the phrase within its source document.


\section*{Acknowledgements}
This work was supported by the IITP grant (No. 2018-0-00584, 2019-0-01906) and the NRF grant (No. 2020R1A2B5B03097210). 
It was also supported by US DARPA KAIROS Program (No. FA8750-19-2-1004), SocialSim Program (No. W911NF-17-C-0099), INCAS Program (No. HR001121C0165), National Science Foundation (IIS-19-56151, IIS17-41317, IIS 17-04532), and the Molecule Maker Lab Institute: An AI Research Institutes program (No. 2019897).

% Entries for the entire Anthology, followed by custom entries
\bibliography{main}
\bibliographystyle{acl_natbib}

\newpage
\appendix
\section{Supplementary Material}
\subsection{Pseudo-code of \proposed}
\label{subsec:pseudocode}
Algorithm~\ref{alg:overview} describes the detailed process of our framework, including the training step (Lines 1--9) and the expansion step (Lines 10--23). 
The final output is the expanded topic taxonomy (Line 24).

\begin{algorithm}
\small
    \DontPrintSemicolon
    \SetKwProg{Fn}{Function}{:}{}
    \SetKwComment{Comment}{$\triangleright$\ }{} 
	%\Comment*[r]{write comment here}
	
	\KwIn{Initial topic taxonomy $\taxo=(\cateset, \edgeset)$ and Text~corpus $\docuset$} 
	\KwOut{Expanded taxonomy $\taxo'$}
	
	\vspace{5pt}
	{\color{blue}{\tcp{Step 1: Learning the topic taxonomy}}}
	$\mathcal{X}\leftarrow$ \textsc{CollectTriples}$(\taxo, \docuset)$ \;
	$\mathcal{G} \leftarrow$ \textsc{ConstructGraph}$(\taxo)$\; 
    \While{not converged}{
    \For{$(\topic{j}, \doc{i}, \phrase{k}) \in \mathcal{X}$}{
        Obtain the model outputs for the inputs $(\mathcal{G}, \topic{j}, \doc{i})$ \;
        %$\topicdocs{c}^* \leftarrow$ \leftarrow RetrieveRelevantDocs($\topicterms{c}, \topicdocs{c};\docuset$)\;
        Compute $\mathcal{L}_{sim}$ by Equation~\eqref{eq:simloss} \;
        Compute $\mathcal{L}_{gen}$ by Equation~\eqref{eq:genloss} \;
        $\mathcal{L} \leftarrow \mathcal{L}_{sim}+\mathcal{L}_{gen}$ \;
        $\Theta \leftarrow \Theta - \eta\cdot {\partial\mathcal{L}}/{\partial\Theta}$ \;
        }
    }

    \vspace{5pt}
    {\color{blue}{\tcp{Step 2: Expanding the topic taxonomy}}}
    $\taxo' \leftarrow \taxo$ \;
    {\color{blue}{\tcp{For each valid position (the~child~position of each topic)}}}
    \For{$\topic{j}\in\cateset$}{
    $\taxo^*, \topicphs{}^* \leftarrow \taxo, \emptyset$ \;
    $\topic{j}^*\leftarrow$ \textsc{MakeVirtualNode}$(\topic{j})$ \;
    $\taxo^*$\textsc{.InsertNode}$(\topic{j}, \{\topic{j}^*\})$ \;
    $\mathcal{G}^* \leftarrow$ \textsc{ConstructGraph}$(\taxo^*)$ \; 
        \For{$\doc{i}\in\docuset$}{
        Obtain the model outputs for the inputs $(\mathcal{G}^*, \topic{j}^*, \doc{i})$ \;
        $\hat{s} \leftarrow \exp(\cvec{j}^{*\top}\bm{M}\dvec{i})$\;
        $\hat{p} \leftarrow [\hat{v}_{1},\ldots,\hat{v}_{T}], \hat{v}_{t}\sim P(\token{t}|\hat{v}_{<t},\doc{i},\vtopic{j})$\;
        
        $\topicphs{}^*$.\textsc{Append}$((\hat{s}, \hat{p}))$\;  
        }
        $\topicphs{}^*\leftarrow$ \textsc{FilterByNormalizedScore}$(\topicphs{}^*, \tau)$ \;
        $\topic{j1}^{*}, \ldots, \topic{jK}^{*} \leftarrow$ \textsc{ClusterPhrases}$(\topicphs{}^*)$ \;
        $\taxo'$.\textsc{InsertNode}$(\topic{j}, \{\topic{j1}^{*}, \ldots, \topic{jK}^{*}\})$ \;
    }

    \Return $\taxo'$
    
\caption{The process of \proposed.}
\label{alg:overview}
\end{algorithm}


\smallsection{Training Step (Lines 1--9)}
\proposed first collects all positive triples $(\topic{j}, \doc{i}, \phrase{k})$ from an initial topic taxonomy $\taxo$ and a text corpus $\docuset$ (Line 1; Section~\ref{subsubsec:training}), and constructs a topic relation graph $\mathcal{G}$ from the topic hierarchy (Line 2; Section~\ref{subsubsec:topic_encoder}).
Then, it updates all the trainable parameters based on the gradient back-propagation (Lines 5--9) to minimize the losses for the topic-document similarity prediction task (Line 6; Section~\ref{subsubsec:prediction}) and the topic-conditional phrase generation task (Line 7; Section~\ref{subsubsec:generation}).

\smallsection{Expansion Step (Lines 10--23)}
Using the trained model, \proposed discovers new topics that need to be inserted into each valid position in the topic hierarchy (Line 11).
For a virtual topic node $\topic{j}^*$ as a newly-introduced child of each topic node $\topic{j}$ (Line 13), it constructs a topic relation graph $\mathcal{G}^*$ from the topic hierarchy augmented with the virtual topic node (Lines 14--15).
Then, it collects all pairs of a topic-document similarity score and a generated topic phrase  $(\hat{s}, \hat{p})$, which are obtained by using the trained model on the augmented topic relation graph and all the documents (Lines 16--20; Section~\ref{subsubsec:collection}).
Next, it filters out non-confident (i.e., irrelevant) phrases according to the normalized score (Line 21), then it performs clustering to find out multiple phrase clusters, each of which is considered as a new topic node having a novel topic semantics (Line 22; Section~\ref{subsubsec:clustering}).
In the end, it inserts the identified new topic nodes into the target position (i.e., the child of a topic node $\topic{j}$) to expand the current topic taxonomy (Line 23).

% \subsection{Reproducibility}
% \label{subsec:reprod}
% For reproducibility, this submission is accompanied by our codes and two datasets (i.e., \amazon and \dbpedia) used in the experiments.
%at the anonymized github repository.\footnote{https://github.com/topicexpan-author/topicexpan}

\begin{figure}[t]
    \centering
    \includegraphics[width=\linewidth]{FIG/phrase_generator.pdf}
    \caption{The phrase generator architecture. It generates the token sequence given a topic and a document, by using topic-attentive token representations as the context.
    }
    \label{fig:phrase_generator}
\end{figure}

\subsection{Baseline Methods}
\label{subsec:basedetail}
For the baselines, we employ the official author codes while following the parameter settings provided by~\cite{lee2022taxocom}.
For all the methods that optimize the Euclidean or spherical embedding space (i.e., \taxogen, \corel, and \taxocom), we fix the number of negative terms (for each positive term pair) to 2 during the optimization.
\begin{itemize}
    \item \textbf{\hlda}\footnote{https://github.com/joewandy/hlda}~\cite{blei2003hierarchical} performs hierarchical latent Dirichlet allocation.
    It models a document generation process as sampling its words along the path selected from the root to a leaf. 
    We set the smoothing parameters $\alpha$ = 0.1 and $\eta$ = 1.0, respectively for document-topic distributions and topic-word distributions, and the concentration parameter in the Chinese restaurant process $\gamma$ = 1.0. 
    
    \item \textbf{\taxogen}\footnote{https://github.com/franticnerd/taxogen}~\cite{zhang2018taxogen} is the unsupervised framework for topic taxonomy construction. 
    To identify hierarchical term clusters, it optimizes the term embedding space with SkipGram~\cite{mikolov2013distributed}.
    %The number of child nodes is manually set, as done in~\cite{zhang2018taxogen,shang2020nettaxo}.
    We set the maximum taxonomy depth to 3 and the number of child nodes to 5, as done in~\cite{zhang2018taxogen,shang2020nettaxo}.
    
    \item \textbf{\corel}\footnote{https://github.com/teapot123/CoRel}~\cite{huang2020corel} is the first topic taxonomy expansion method.
    It trains a topic relation classifier by using the initial taxonomy, then recursively transfers the relation to find out candidate terms for novel subtopics. 
    Finally, it identifies novel topic nodes based on term embeddings induced by SkipGram~\cite{mikolov2013distributed}.
    
    \item \textbf{\taxocom}\footnote{https://github.com/donalee/taxocom}~\cite{lee2022taxocom} is the state-of-the-art method for topic taxonomy expansion. 
    For each node from the root to the leaf, it recursively optimizes term embedding and performs term clustering to identify both known and novel subtopics.
    we set $\beta=1.5, 2.5, 3.0$ (for each level) in the novelty threshold $\tau_{nov}$, and fix the signficance threshold $\tau_{sig}=0.3$.
    
    % \item \textbf{\proposed}: The proposed framework for topic taxonomy expansion.
    % It trains a topic-conditional phrase generator by using the initial topic taxonomy, then generates topic-related phrases for a virtual topic node whose semantics is captured based on its surrounding relation structure in the hierarchy.
\end{itemize}

\subsection{Implementation Details}
\label{subsec:implementation}
% \subsubsection{The \proposed Framework}
\smallsection{Model Architecture}
For the topic encoder, we use two GCN layers to avoid the over-smoothing problem, and fix the dimensionality of all node representations to 300.
For the document encoder, we employ the \texttt{bert-base-uncased} provided by huggingface~\cite{devlin2019bert}, as the initial checkpoint of a pretrained model.
It contains 12 layers of transformer blocks with 12 attention heads, thereby obtaining 768-dimensional contextualized token representations $[\vvec{i1}, \ldots, \vvec{iL}]$ (and a final document representation $\dvec{i}=\text{mean-pooling}(\vvec{i1}, \ldots, \vvec{iL})$) for an input document $\doc{i}$.
Consequently, the size of the interaction matrix $\bm{M}$ in our topic-document similarity predictor (Equation~\eqref{eq:simloss}) becomes 300 $\times$ 768.
For the phrase generator, we adopt a single layer of the transformer decoder with 16 attention heads\footnote{We empirically found that the number of decoding layers hardly affects the performance (i.e., accuracy) of topic-conditional phrase generation.} and train its parameters from scratch without using the checkpoint of a pretrained text decoder.
We limit the maximum length of a generated phrase to 10.
Figure~\ref{fig:phrase_generator} shows the phrase generator architecture.
In total, our neural model contains 540K (for the topic encoder), 110M (for the document encoder), 230K (for the similarity predictor), and  30M (for the phrase generator) parameters.

% \footnote{We adopt the greedy strategy, which selects the word token of the max probability.}
%\footnote{Two special tokens, \texttt{[BOP]} and \texttt{[EOP]}, indicate the begin and the end of each phrase.}
%\footnote{For training, we take the previous tokens from gold standards (i.e., target phrases) as the input token sequence of the transformer decoder, also known as \textit{teacher forcing}.}
% \footnote{For inference, we use the previously predicted tokens as the input of the decoder.}

\smallsection{Training Step}
For the optimization of model parameters, we use the Adam optimizer~\cite{kingma2014adam} with the initial learning rate 5e-5 and the weight decay 5e-6.
The batch size is set to 64, and the temperature parameter $\gamma$ in Equation~\eqref{eq:simloss} is set to 0.1.
The best model is chosen using the best perplexity of generated topic phrases on the validation set of positive triples $(\topic{j}, \doc{i}, \phrase{k})$, which is evaluated every epoch.

\smallsection{Expansion Step}
To filter out non-confident phrases (Section~\ref{subsubsec:collection}), we set the threshold value $\tau$ to 0.8 after applying min-max normalization on all topic-document similarity scores computed for each virtual topic node.
To perform $k$-means clustering on the collected topic phrases (Section~\ref{subsubsec:clustering}), we set the initial number of clusters $k$ to 10, then select top-5 clusters by their cluster size (i.e., the number of phrases assigned to each cluster).
The center phrase of each cluster is used as the final topic name of the new topic node.

\subsection{Computing Platform}
All the experiments are carried out on a Linux server machine with Intel Xeon Gold 6130 CPU @2.10GHz and 128GB RAM by using a single RTX3090 GPU.
In this environment, the model training of \proposed takes around 2 hours and 6 hours for \amazon and \dbpedia, respectively.

\begin{table}[b]
\small
\caption{Three disjoint parts of the topic taxonomy.}
\label{tbl:taxopart}
\centering
\resizebox{0.99\linewidth}{!}{%
\begin{tabular}{VcT}
    \toprule
        \textbf{Corpus} & \textbf{Part} & \textbf{First-level topics} \\\midrule
        & $\subtaxo{1}$ & grocery gourmet food, toys games\\
        \amazon & $\subtaxo{2}$ & beauty, personal care \\
        & $\subtaxo{3}$ & baby products, pet supplies \\\midrule
        & $\subtaxo{1}$ & agent, work, place \\
        \dbpedia & $\subtaxo{2}$ & species, unit of work, event \\
        & $\subtaxo{3}$ & sports season, device, topical concept \\
    \bottomrule
\end{tabular}
}
\end{table}

\subsection{Quantitative Evaluation Protocol}
\label{subsec:evalprotocol}
For exhaustive evaluation on a large-scale topic taxonomy with hundreds of topic nodes, the output taxonomy of topic taxonomy expansion methods (i.e., \corel, \taxocom, and \proposed) is divided into three parts $\subtaxo{1}$, $\subtaxo{2}$, and $\subtaxo{3}$ so that each part covers some of the first-level topics (and their subtrees) listed in Table~\ref{tbl:taxopart}.

In case of \hlda and \taxogen, the first-level topics in their output taxonomies are not matched with the ground-truth topics (in Table~\ref{tbl:taxopart}), because they build a topic taxonomy from scratch.
For this reason, in Table~\ref{tbl:humaneval}, their output taxonomies are evaluated whole without partitioning.
In addition, the two metrics for novel topic discovery (i.e., relation accuracy and subtopic integrity) are designed to evaluate the topic taxonomy expansion methods, so it is infeasible to measure the aspects on the output taxonomies of \hlda and \taxogen.
Thus, we only report the metric for topic identification (i.e., term coherence) in Table~\ref{tbl:humaneval}.

\smallsection{Term Coherence}
It indicates how strongly terms in a topic node are relevant to each other. 
Evaluators count the number of terms that are relevant to the common topic (or topic name) among the top-5 terms found for each topic node.

\smallsection{Relation Accuracy}
It computes how accurately a topic node is inserted into a given topic hierarchy (i.e., \textit{precision} for novel topic discovery).
For each valid position, evaluators count the number of newly-inserted topics that are in the correct relationship with the surrounding topics.

\smallsection{Subtopic Integrity} 
It measures the completeness of subtopics for each topic node (i.e., \textit{recall} for novel topic discovery).
Evaluators investigate how many ground-truth novel topics, which were deleted from the original taxonomy, match with one of the newly-inserted topics.


\subsection{Examples of Topic Phrase Generation}
\label{subsec:examples}
We provide additional examples of topic-conditional phrase generation, obtained by \proposed.
Figure~\ref{fig:examples} illustrates a confident phrase (Left) and a non-confident phrase (Right), generated from each input document and the given relation structure of a target topic, for both datasets.
As discussed in Section~\ref{subsubsec:casestudy}, 
in case that a target topic is relevant to the document (i.e., high topic-document similarity score), \proposed successfully generates a phrase relevant to the target topic.
On the other hand, in case that a target topic is irrelevant to the document (i.e., low topic-document similarity score), \proposed obtains a phrase irrelevant to the target topic.

\begin{figure}[t]
\centering
\begin{subfigure}{\linewidth}
    \centering
    \includegraphics[width=\linewidth]{FIG/case_amazon.pdf}  
    \caption{Dataset: \amazon}
\end{subfigure}
\begin{subfigure}{\linewidth}
    \centering
    \includegraphics[width=\linewidth]{FIG/case_dbpedia.pdf}
    \caption{Dataset: \dbpedia}
\end{subfigure}
\caption{Examples of topic-conditional phrase generation, given a document and its relevant/irrelevant topic.}
\label{fig:examples}
\end{figure}

\end{document}
